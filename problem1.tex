\section{Constructing King models}
In class, I motivated the King model, which has the distribution function described by the dimensionless potential
\begin{equation}
    f\propto e^W -1
\end{equation}
where 
\begin{equation}
    W(r)\equiv\frac{V(r_t)-V(r)}{\sigma^2}
\end{equation}
Recall that the profile shape only depends on Wo ≡ W(0). Given the
profile, the overall linear size and mass of the profile can be scaled as
desired to any set of units
\subsection{}
Note that Poisson’s equation is a second order ordinary differential
equation just like Newton’s equations and therefore can be written as two coupled first order differential equations. Rewrite the
Poisson equation as a system of first order ODEs with “physical
boundary conditions” (that is: $V(0) = $ constant, $dV(0)/dr = 0$) in
spherical coordinates.

\subsection{}
Integrate (numerically) Poisson’s equation for the King model. By
adjusting $W_o\equiv W(0)$, one gets models with different concentration; you can check against the table below. Plot up the density and the potential just to see what they look like. You will
use these later on to generate initial conditions for a simulation.
Check your procedure by computing the values in each of the
four columns. The core radius of the King model is defined as
\begin{equation}
    r_c=\sqrt{9\sigma^2/4\pi G \rho_0}.
\end{equation}
In these units, the second column describes
the total energy of the King model, the third column describes the
concentration, c, the fourth is the the central density in units of
the mean density and the fifth is the mean square particle radius in
units of the tidal radius.

\begin{table}[]
    \centering
    \include{KingModel.tex}
    \caption{Caption}
    \label{tab:my_label}
\end{table}




% % Show that in a frame that rotates with constant angular velocity $\mathbf{\Omega}$,
% % with $\nabla\mathbf{\Phi}_{eff}\equiv\mathbf{\Phi}-\frac{1}{2}|\mathbf{\Omega}\times\mathbf{x}|^2$, the collisionless Boltzmann equation can
% % be written:
% % \begin{equation}
% %     \frac{\partial f}{\partial t}+\mathbf{v}\cdot\nabla f - \left[2(\mathbf{\Omega}\times\mathbf{v})+\nabla\mathbf{\Phi}_{eff}\right]\cdot\frac{\partial f}{\partial \mathbf{v}}=0
% % \end{equation}
% % [Hint: This is B\&T, Problem 4.1 on page 387.]

% The continuity equation is
% \begin{equation}
%     \frac{\partial f}{\partial t} + \frac{\partial}{\partial w}\cdot (f\Dot{w})=0
% \end{equation}
% where we can consider $w=(p,q)$ as any arbitrary system of canonical coordinates \cite{BT2008} and $\Dot{w}\mathrm{is}(\Dot{p},\Dot{q})$. 
 
% In a rotating frame, the canonical coordinates are $w=(x,v)$ and $\Dot{w}=(\Dot{x},\Dot{v})$ so the continuity equation is given by
% \begin{align*}
%     &\frac{\partial f}{\partial t} + \frac{\partial}{\partial \mathbf{w}}\cdot (f\Dot{\mathbf{w}})=0\\
%     &\frac{\partial f}{\partial t} + \frac{\partial}{\partial \mathbf{x}}\cdot (f\Dot{\mathbf{x}})+ \frac{\partial}{\partial \mathbf{v}}\cdot (f\Dot{\mathbf{v}})=0\\
%     &\frac{\partial f}{\partial t} + \frac{\partial f}{\partial \mathbf{x}}\cdot \Dot{\mathbf{x}} +\frac{\partial \dot{\mathbf{x}}}{\partial \mathbf{x}}\cdot f + \frac{\partial f}{\partial \mathbf{v}}\cdot \Dot{\mathbf{v}} + \frac{\partial \Dot{\mathbf{v}}}{\partial \mathbf{v}}\cdot f=0\\
%     &\frac{\partial f}{\partial t} + \nabla f\cdot\Dot{\mathbf{x}} + \frac{\partial f}{\partial \mathbf{v}}\cdot \Dot{\mathbf{v}}=0.\numberthis{\label{eq:cont}}
% \end{align*}
% where 
% \begin{align*}
%     \mathbf{v}&=\dot{\mathbf{x}}\\
%     \dot{\mathbf{v}}&=\Ddot{\mathbf{x}}=-\nabla\mathbf{\Phi} - 2\mathbf{\Omega}\times\dot{\mathbf{x}}-\mathbf{\Omega}\times(\mathbf{\Omega}\times\mathbf{x})\\
%     &= -\nabla\mathbf{\Phi}_{eff} - 2\mathbf{\Omega}\times\dot{\mathbf{v}}
% \end{align*}

% Substituting the last equations into the continuity equation \ref{eq:cont} and  we get
% \begin{align*}
%     &\frac{\partial f}{\partial t} + \nabla f\cdot \mathbf{v} + \frac{\partial f}{\partial \mathbf{v}}\cdot (-\nabla\Phi_{eff} - 2\mathbf{\Omega}\times\dot{\mathbf{x}}) =0\\
%     &\frac{\partial f}{\partial t}+\mathbf{v}\cdot\nabla f - \left(2\mathbf{\Omega}\times\mathbf{v}+\nabla\mathbf{\Phi}_{eff}\right)\cdot\frac{\partial f}{\partial \mathbf{v}}=0
% \end{align*}



