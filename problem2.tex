\section{Models with non-isotropic distribution functions}

In class, we discussed
models with $f=f(E)$ One feature of isotropic models is the that velocity ellipsoid is spherical. Many times one needs models where this
is not the case. Here is practice with a few simple non-isotropic models.

\subsection{}

% Show that
% \begin{equation}
%     \frac{v^{-2}_\theta}{v^{-2}_r}=1-\beta
% \end{equation}
% for the following distribution function:
% \begin{equation}
%     f(E,L)=L^{-2\beta}g(E).
% \end{equation}
% There are many acceptable and useful choices for isotropic part
% of the distribution, g(E) (e.g. one could choose it to be a King
% model). [This is similar to B\&T, Problem 4.6, pg. 387 (2nd edition).]
An anisotropic distribution can be given by
\begin{equation}
    f(E,L)=L^{-2\beta}g(E).
\end{equation} 
where g(E) is the isotropic part of the distribution. 

We can choose that g(E) will be a Maxwellian distributed model described by,
\begin{equation}
    g(E)=g_0e^{-v_r/2\sigma^2}e^{-v_t^2/2\sigma^2}
    \label{eq:maxModel}
\end{equation}
where we have use $v^2=v_r^2+v_t^2$, so that the anisotropic distribution will be
\begin{equation}
    f(E,L)=(rv_t)^{-2\beta}g_0e^{-v_r/2\sigma^2}e^{-v_t^2/2\sigma^2},
\end{equation}
assuming that the potential $\Phi(x)$ is spherically symmetric so that the angular momentum is given by 
\begin{equation}
    L=rv_t.
    \label{eq:angMom}
\end{equation}

Then, we can calculate the ratio $v^{-2}_\theta/v^{-2}_r$ using the following,
\begin{align}
    \Bar{v_r}^2=\frac{2\pi}{\nu}\int_{-\infty}^{\infty} dv_r v_r^2 \int_{0}^{\infty} dv_t v_t f(E,L)\\
    \Bar{v_\theta}^2=\frac{\pi}{\nu}\int_{0}^{\infty} dv_t v_t^3 \int_{-\infty}^{\infty} dv_r f(E,L)
\end{align}
and solving for $\Bar{v_r}^2$ and $\Bar{v_\theta}^2$.
First, we will solve for $\Bar{v_r}^2$,
\begin{align*}
    \Bar{v_r}^2=\frac{2\pi}{\nu}\int_{-\infty}^{\infty} dv_r v_r^2 \int_{0}^{\infty} dv_t v_t (rv_t)^{-2\beta}g_0e^{-v_r^2/2\sigma^2}e^{-v_t^2/2\sigma^2}\\
    =\frac{2\pi}{\nu}g_0r^{-2\beta}e^{-1/\sigma^2}\int_{-\infty}^{\infty} dv_rv_r^2e^{-v_r^2}\int_{0}^{\infty} dv_t v_t^{1-2\beta}e^{-v_t^2}\\
\end{align*}
Now that the integrals don't depend of each other, so we can use the identities for Gamma functions,
\begin{align}
    \Gamma(n+1)&=n\Gamma(n),\\
    \int_{-\infty}^{\infty}dx x^ne^{-x^2}&=\Gamma\left(\frac{n+1}{2}\right), \text{ for even functions}\\
    \int_{0}^{\infty}dx x^ne^{-x^2}&=\frac{1}{2}\Gamma\left(\frac{n+1}{2}\right)\\
\end{align}
Then,
\begin{align*}
     \Bar{v_r}^2&=\frac{2\pi}{\nu}g_0r^{-2\beta}e^{-1/\sigma^2}\Gamma\left(\frac{3}{2}\right)\cdot\frac{1}{2}\Gamma\left(\frac{2+2\beta}{2}\right)\\
     &=\frac{2\pi}{\nu}g_0r^{-2\beta}e^{-1/\sigma^2}\frac{1}{2}\sqrt{\pi}\frac{1}{2}\Gamma\left(1+\beta\right)\\
     &=\frac{\pi}{2\nu}g_0r^{-2\beta}e^{-1/\sigma^2}\sqrt{\pi}\Gamma\left(1+\beta\right)
\end{align*}

Likewise we can solve for $\Bar{v_\theta}^2$,
\begin{align*}
    \Bar{v_\theta}^2&=\frac{\pi}{\nu}\int_{0}^{\infty} dv_t v_t^3 \int_{-\infty}^{\infty} dv_r (rv_t)^{-2\beta}g_0e^{-v_r/2\sigma^2}e^{-v_t^2/2\sigma^2}\\
    &=\frac{\pi}{\nu}g_0r^{-2\beta}e^{-1/\sigma^2}\int_{0}^{\infty} dv_t v_t^{3-2\beta}e^{-v_t^2} \int_{-\infty}^{\infty} dv_re^{-v_r}\\
    &=\frac{\pi}{\nu}g_0r^{-2\beta}e^{-1/\sigma^2}\frac{1}{2}\Gamma\left(\frac{4+2\beta}{2}\right)\Gamma\left(\frac{1}{2}\right)\\
    &=\frac{\pi}{\nu}g_0r^{-2\beta}e^{-1/\sigma^2}\frac{1}{2}(1-\beta)\Gamma\left(1-\beta\right)\sqrt{\pi}\\
    &=\frac{\pi}{2\nu}g_0r^{-2\beta}e^{-1/\sigma^2}\sqrt{\pi}(1-\beta)\Gamma\left(1-\beta\right)
\end{align*}

Finally, the ratio is
\begin{align*}
    \frac{v^{-2}_\theta}{v^{-2}_r}&=\frac{\frac{\pi}{2\nu}g_0r^{-2\beta}e^{-1/\sigma^2}\sqrt{\pi}\Gamma\left(1+\beta\right)}{\frac{\pi}{2\nu}g_0r^{-2\beta}e^{-1/\sigma^2}\sqrt{\pi}(1-\beta)\Gamma\left(1-\beta\right)}\\
    \frac{v^{-2}_\theta}{v^{-2}_r}&=1-\beta\\
\end{align*}


\subsection{}
% Show that the distribution function
% \begin{equation}
%     f(E,L) = \left\{\begin{array}{lr}
%         f_0\delta(L^2)(E_0-E)^{-1/2} & \text{for } E< E_0\\
%         0, & \text{otherwise }
%         \end{array}\right.
% \end{equation}

% where $f_0$ and $E_0$ are constants, self-consistently generate a model
% with density
% \begin{equation}
%         \rho(r) = \left\{\begin{array}{lr}
%         C/r^2 & \text{for } r< r_0\\
%         0, & \text{otherwise }
%         \end{array}\right.
% \end{equation}

% where $C$ is some constant and $r_o$ satisfies $V(r_o) = E_o$ (similar to
% the tidal radius in the King models). Because of the Dirac delta
% function in $L^2$, this model consists of radial orbits (and is unstable
% as we will discuss later on . . . ). [This is B\&T, Problem 4.9, pg.
% 388 (2nd edition); note the different conventions for the sign of E
% than my typical usage in class.]

The density is given by $\rho=\int d^3v f(E,L)$ where $d^3v=2\pi dv_r dv_t v_t$ for spherical coordinates so given a distribution function of the form
\begin{equation}
    f(E,L) = \left\{\begin{array}{lr}
        f_0\delta(L^2)(E_0-E)^{-1/2} & \text{for } E< E_0\\
        0, & \text{otherwise }
        \end{array}\right.
\end{equation}
we can find that the density is
\begin{align*}
    \rho &=\int_\mathbf{v} 2\pi dv_r dv_t v_t f_0\delta(L^2)(E_0-E)^{-1/2}\\
    &= 2\pi \int dv_r \int dv_t v_t f_0\delta(L^2)(E_0-E)^{-1/2}\\
\end{align*}
We can leave the second integral in terms of $L^2$ by using equation \ref{eq:angMom} so that
\begin{align*}
    v_t^2=L^2/r^2 \text{ and } v_tdv_t=dL^2/2r^2.
\end{align*}
We also know that the energy is
\begin{equation}
    E=\frac{v_r^2+v_t^2}{2}+V(r)=\frac{v_r^2}{2}+\frac{L^2}{r^2} + V(r)
\end{equation}
Substituting all of these we get 
\begin{align*}
    \rho & = 2\pi \int dv_r \int \frac{dL^2}{2r^2} f_0 \delta(L^2) (E_0 - \left(\frac{v_r^2}{2}+\frac{L^2}{r^2} + V(r)\right))^{-1/2}\\
    &= \frac{\pi f_0}{r^2}\int dv_r \int dL^2 \delta(L^2) (E_0 - \left(\frac{v_r^2}{2}+\frac{L^2}{r^2} + V(r)\right))^{-1/2}\\
    &= \frac{\pi f_0}{r^2}\int dv_r \left(E_0 - \left(\frac{v_r^2}{2} + V(r)\right)\right)^{-1/2}\\
    &= \frac{\pi f_0}{r^2} \left[ \sqrt{2} \tan^{-1}\left( \frac{v_r}{(E_0-V(r)-v^2_r)}\right)\right]^{v_{max}}_0\\
    &= \frac{\pi f_0}{r^2}\sqrt{2}\left[ \tan^{-1}\left(\infty\right) - \tan^{-1}\left(0\right)\right]\\
    &= \frac{\pi f_0}{r^2}\sqrt{2}\frac{\pi}{2}=\frac{\pi^2f_0\sqrt{2}}{2}\frac{1}{r^2}\\
    &= \frac{C}{r^2}
\end{align*}
where we had used the following identities
\begin{align}
    \int^a_b dx \delta(x-c)f(x)=f(c)\\
    \int\frac{dx}{\sqrt{a^2-x^2}}=\tan^{-1}\left(\frac{1}{\sqrt{a^2-x^2}}\right)
\end{align}
and $v_{max}$ is when $E=E_0$ so $v_{max}=\sqrt{2(E_0-V(r))}$


